\documentclass[../main.tex]{subfiles}

\begin{document}

\chapter{June $19^{th} / 2025$}
\label{ch:tufte-design}

\section{Next-Generation Sequencing} \cite{gencore_how_sequencing_works}

\begin{itemize}
    \item \textbf{Read:} Single sequence produced from a sequencer.
    \item \textbf{Library:} Collection of DNA fragments that have been prepared for sequencing.
    \item \textbf{Flowcell:} Chip on which DNA is loaded and provided to the sequencer.
    \item \textbf{Lane:} Open portion of a flowcell. Usually for technical replicates or different samples.
    \item \textbf{Run:} Entire sequencing reaction from start to finish.
\end{itemize}

\subsection*{Steps:} % Using \subsection* for an unnumbered heading
\begin{enumerate}
    \item Sample collection/preparation
    \item Amplification
    \item Basecalling
\end{enumerate}

\begin{center} % Centering the TikZ picture
\begin{tikzpicture}[ % Start TikZ environment
    every node/.style={draw, circle, fill=blue!20, minimum size=1cm, inner sep=1pt}, % Adjusted minimum size for better text fit
    grandchildnode/.style={draw, rectangle, fill=green!20, text width=3cm, align=center, minimum height=1.5cm}, % Centered text in grandchild
    level 1/.style={sibling distance=4cm, level distance=2.5cm}, % Distance for first level children
    level 2/.style={sibling distance=2cm, level distance=3.5cm}  % Distance for grandchildren
]
\node {Sequencing}
    child {node {Single-Read}
        child {node[grandchildnode] {
            - Simple library prep \\
            - Low input DNA \\
            - Economical
        }}
    }
    child {node {Paired-End}
        child {node[grandchildnode] {
            - Simple paired-end libraries \\
            - Efficient sample use \\
            - Broad range applications
        }}
    }; % Semicolon terminates the \node command
\end{tikzpicture} % End TikZ environment
\end{center}

\section{File Formats in Genomics}

\subsection*{FastA Format:} Most basic for reporting a sequence. Contains sequence name, description.
\begin{verbatim}
>Chr1 CHROMOSOME dumped from ADB:
Jun/20/09 14:53; last updated: 2009-02-02
CCCTAAACCCTAAACCCTAAACCCTAAACCTC
TGAATCCTTAATCCCTAAATCCCTAAATCTTT
AAATCCTACATCCAT
\end{verbatim}
\textit{DB query tools} such as \textbf{blast} and \textbf{multiple-sequence alignment} programs accept only FastA. Reference Genomes are often delivered in this format.

\subsection*{FastQ Format:} Most widely used in sequence analysis. Output delivered from a sequencer. More information is contained in this format.
\begin{verbatim}
@SEQ_ID
GATTTGGGGTTCAAAGCAGTATCGATCAAATAGTAAATCCAT
TTGTTCAACTCACAGTTT
+
!''*((((***+))%%%++)(%%%%).1***-+*''))**5
5CCF>>>>>>CCCCCCC65
\end{verbatim}
\textit{Quality value characters. Lowest: \texttt{!} and highest: \texttt{\textasciitilde{}}}:
\begin{verbatim}
!"#$%&'()*+,-./0123456789:;<=>?@ABCDEFGHIJKLMN
OPQRSTUVWXYZ[\]^_`abcdefghijklmnopqrstuv
wxyz{|}~
\end{verbatim}

\begin{itemize}
    \item Sequence Header:
    \begin{itemize}
        \item \texttt{@} is sequence identifier
        \item rest is sequence description
    \end{itemize}
    \item Second line is sequence
    \item Third line starts with \texttt{+} and can have same sequence identifier appended
    \item Fourth are quality scores (ASCII encoded)
\end{itemize}
Sequence Header contains: Instrument name, run ID, flowcell ID, flowcell lane, tile number, X and Y coordinates of cluster, member of a pair, filtered status, control bits, index sequence, etc.

Nearly everything works with FastQ except for: Blast, Multiple sequence alignment (typically require FastA), any reference sequence (usually FastA).

\subsection*{Quality Scores/Q-score:} Integer value assigned to each nucleotide base. Represents estimated probability ($P$) that base call is incorrect. Logarithmic scale: $Q = -10\log_{10}(P)$.
\textbf{Higher Q-score} = Higher confidence = lower $P$ (error) = higher accuracy.
\textbf{Lower Q-score} = Lower confidence = higher $P$ (error).

\textbf{Examples:}
\begin{itemize}
    \item \textbf{Q10:} $P=0.1$ (1 in 10 error). Accuracy = $90\%$.
    \item \textbf{Q20:} $P=0.01$ (1 in 100 error). Accuracy = $99\%$. Often a minimum acceptable quality.
    \item \textbf{Q30:} $P=0.001$ (1 in 1,000 error). Accuracy = $99.9\%$. Benchmark for high-quality.
    \item \textbf{Q40:} $P=0.0001$ (1 in 10,000 error). Accuracy = $99.99\%$.
\end{itemize}

Common uses: filter bases/reads if a threshold is not met.
Main purpose: provide evidence that the sequence, alignment, assembly, SNP are real and not sequencing artifacts.

\subsection*{SAM Format:} Sequence Alignment/Map. Basic, human-readable text format. Generated by most alignment algorithms. Consists of:
\begin{itemize}
    \item \textbf{Header section (optional):} Lines start with \texttt{@}. Contains metadata: SAM format version (\texttt{@HD}), reference sequence dictionary (\texttt{@SQ}), read groups (\texttt{@RG}), program used (\texttt{@PG}), comments (\texttt{@CO}).
    \item \textbf{Alignment section:} Each line is an alignment record for a single read. Contains 11 mandatory fields, followed by optional fields. \\
    \textbf{Field Descriptions:}
    \begin{enumerate}
        \item QNAME: Query template NAME
        \item FLAG: bitwise FLAG
        \item RNAME: Reference sequence NAME
        \item POS: 1-based leftmost mapping POSition
        \item MAPQ: MAPping Quality
        \item CIGAR: CIGAR string
        \item RNEXT: Ref. name of the mate/next read
        \item PNEXT: Position of the mate/next read
        \item TLEN: observed Template LENgth
        \item SEQ: segment SEQuence
        \item UAL: ASCII of Phred-scaled base QUALity+33
    \end{enumerate}
\end{itemize}

\subsection*{BAM:} Same format except that it is encoded in binary (faster to read) but not human legible.

\subsection*{CRAM:} Retains same info as SAM and is compressed in more efficient way.

Formats are \textit{output} from aligners and assemblers.

\subsection*{BED Format:} Simple way to define basic sequence features to a sequence. One line per feature, each containing 3 - 12 columns of data plus optional track definition lines. Generally used for user defined sequence features as well as graphical representations of features.

\begin{itemize}
    \item Chromosome Name
    \item Chromosome Start
    \item Chromosome End
\end{itemize}

\textbf{Optional Fields}
Nine additional fields are optional for feature definition. If higher-numbered optional fields are used, all lower-numbered fields preceding them must also be populated.

\begin{description}
    \item[Name] Label to be displayed under the feature if enabled in the page configuration.
    \item[Score] A numerical score ranging from 0 to 1000. The display style for scored data can be configured using track lines (see below).
    \item[Strand] Defines the orientation of the feature: `+` (forward) or `–` (reverse).
    \item[thickStart] The coordinate where the feature representation begins as a solid rectangle.
    \item[thickEnd] The coordinate where the feature representation as a solid rectangle ends.
    \item[itemRgb] An RGB color value (e.g., 0,0,255). This is applied only if a track line specifies \texttt{itemRgb="on"} (case-insensitive).
    \item[blockCount] The number of sub-elements (e.g., exons) within the feature.
    \item[blockSize] A comma-separated list of the sizes of these sub-elements.
    \item[blockStarts] A comma-separated list of the start coordinates for each sub-element, relative to the feature's start coordinate.
\end{description}

\textbf{Track Lines}
Track definition lines configure the display of features, such as grouping them into separate tracks. Track lines must precede the list of features they affect and consist of the word \texttt{track} followed by space-separated \texttt{key=value} pairs. Valid parameters for Ensembl include:

\begin{description}
    \item[name] A unique identifier for the track when parsing the file.
    \item[description] A label displayed under the track in detailed views (e.g., "Region in Detail").
    \item[priority] An integer determining the display order if multiple tracks are defined.
    \item[color] Specified as RGB, hexadecimal, or an \href{https://www.X.org/releases/X11R7.6/doc/xorg-docs/specs/RGBColorNames.txt}{X11 named color}.
    \item[useScore] A value from 1 to 4, dictating how scored data is displayed. May require additional parameters:
    \begin{itemize}
        \item Tiling array
        \item Colour gradient (defaults to Yellow-Green-Blue with 20 grades). Optionally, custom colors (\texttt{cgColour1}, \texttt{cgColour2}, \texttt{cgColour3}) and the number of grades (\texttt{cgGrades}) can be specified.
        \item Histogram
        \item Wiggle plot
    \end{itemize}
    \item[itemRgb] If set to \texttt{on} (case-insensitive), the individual RGB values defined for each feature (in the \texttt{itemRgb} field) will be used.
\end{description}

\subsection*{BedGraph Format}
The BedGraph format is designed for displaying moderate amounts of scored data and is based on the BED format with these key differences:
\begin{itemize}
    \item The score is located in column 4 (instead of column 5 as in standard BED with score).
    \item Track lines are \textbf{compulsory} and must include \texttt{type=bedGraph}.
\end{itemize}
Optional track line parameters currently supported by Ensembl for BedGraph are:
\begin{description}
    \item[name] (as described above)
    \item[description] (as described above)
    \item[priority] (as described above)
    \item[graphType] Specifies the display style, either \texttt{bar} or \texttt{points}.
\end{description}



\end{document}