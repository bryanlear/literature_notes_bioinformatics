\documentclass[../main.tex]{subfiles}

\begin{document}

\chapter{June $25^{th} / 2025$}
\label{ch:tufte-design}

\section{Simple Conceptual Pipeline }
\hrulefill

% For best practice, the \usetikzlibrary and \tikzstyle commands should be in the preamble
% of your main.tex file. However, they are kept here to match your provided structure.
\usetikzlibrary{arrows.meta, positioning, shapes.geometric, fit}

\begin{figure}[htbp]
    \centering
    
    \tikzstyle{startend} = [rectangle, rounded corners, minimum width=3cm, minimum height=1cm, text centered, draw=black, fill=gray!20]
    \tikzstyle{process} = [rectangle, minimum width=3cm, minimum height=1cm, text centered, text width=3cm, draw=black, fill=blue!20]
    \tikzstyle{io} = [trapezium, trapezium left angle=70, trapezium right angle=110, minimum width=3cm, minimum height=1cm, text centered, draw=black, fill=orange!20]
    \tikzstyle{db} = [cylinder, shape border rotate=90, aspect=0.2, minimum height=1.5cm, text width=2.5cm, text centered, draw=black, fill=green!20]
    \tikzstyle{decision} = [diamond, aspect=2, minimum width=3cm, minimum height=1cm, text centered, draw=black, fill=yellow!20]
    \tikzstyle{report} = [rectangle, rounded corners, minimum width=3cm, minimum height=1.5cm, text centered, draw=black, fill=red!20, text width=3.5cm]
    \tikzstyle{arrow} = [thick, ->, >=Stealth]
    \tikzstyle{group} = [rectangle, rounded corners, draw=black!50, dashed, inner sep=10pt]

    % Resize the whole tikz picture to fit the text width
    \resizebox{\textwidth}{!}{
        \begin{tikzpicture}[node distance=2cm and 1.5cm]
        
        % Nodes
        \node (start) [startend] {Raw Sequencing Data (FASTQ)};
        \node (qc) [process, below=of start] {1. Quality Control (FastQC)};
        \node (align) [process, below=of qc] {2. Alignment to hg38 (BWA-MEM / DeepVariant)};
        \node (postproc) [process, below=of align] {3. Post-processing (Sort, Index, Mark Duplicates)};
        \node (call) [process, below=of postproc] {4. Variant Calling (HaplotypeCaller)};
        \node (annotate) [process, below=of call] {5. Variant Annotation (VEP / SnpEff)};
        
        % Databases for Annotation
        \node (gnomad) [db, left=of annotate, xshift=-2cm] {gnomAD};
        \node (clinvar) [db, right=of annotate, xshift=2cm] {ClinVar};
        
        \draw [arrow] (gnomad) -- (annotate);
        \draw [arrow] (clinvar) -- (annotate);
        
        % Core AI & Bayes Module
        \node (bayes) [decision, below=of annotate, yshift=-1.5cm] {6. Bayesian VUS Assessment \\ $P(\text{Pathogenic} | \text{Data}, G)$};
        
        % Knowledge Graph Construction
        \node (kg_pheno) [io, left=of bayes, xshift=-4cm] {Patient Phenotypes (HPO)};
        \node (kg_ppi) [db, above=of kg_pheno] {STRING-DB};
        \node (kg_gwas) [db, below=of kg_pheno] {GWAS Catalog / OMIM};
        \node (kg_lit) [db, below=of kg_gwas] {PubMed (NLP for relations)};
        
        % Group for KG
        \node[group, fit=(kg_pheno) (kg_ppi) (kg_gwas) (kg_lit), label={[xshift=1cm, yshift=-0.2cm]7. Cardiovascular KG Construction}] (kg_group) {};
        
        % Graph Neural Network
        \node (gnn) [process, right=of bayes, xshift=4cm, fill=purple!20] {8. GNN Pathogenicity Prediction};
        
        % Final Report
        \node (report) [report, below=of bayes, yshift=-1.5cm] {9. AI-Generated Clinical Report \\ (Variant, $P(\text{Patho.})$, KG Evidence, XAI Explanation)};
        
        % Arrows
        \draw [arrow] (start) -- (qc);
        \draw [arrow] (qc) -- (align);
        \draw [arrow] (align) -- (postproc);
        \draw [arrow] (postproc) -- (call);
        \draw [arrow] (call) -- (annotate);
        \draw [arrow] (annotate) -- (bayes);
        \draw [arrow] (kg_group) -- (bayes);
        \draw [arrow] (kg_group) -- (gnn);
        \draw [arrow] (annotate) -- (gnn);
        \draw [arrow] (bayes) -- (report);
        \draw [arrow] (gnn) -- (report);
        
        \end{tikzpicture}
    }
    \caption{Conceptual pipeline example.}
    \label{fig:dummy-pipeline}
\end{figure}


\section{Dataflow Programming} Paradigm that models a program as a directed graph of data flowing between operations. Specify independent tasks and the data dependencies between them, The program excecutes as data becomes available. 

\subsection{Components}
\begin{itemize}
    \item \textbf{Processes}: Operations/ Nodes in graph. Self-contained independent script that performs x computational task. (e.g., command-line tool, Python / R scripts,...,)
    \item\textbf{Channels}: Data Pipes / Edges. Asynchronous queues that connect processes. (sort of like a NN block or something like that).
\end{itemize}

\subsection{Flow Data}
Driven entirely by the availability of data channels.
\begin{itemize}
  
    \item \textbf{Initiation}:When $\geq 1$ initial channels are populated. Channel Factory ``Channel.somepath``
    \item \textbf{Process Execution}  
    \item \textbf{Implicit Parallelism}
\end{itemize}

\textbf{e.g., ``samtools`` pipeline}

\begin{verbatim}
Input Channel (e.g., a BAM file)
      |
      V
[ samtools_sort ]
      |
      +-----------------+-----------------+
      |                 |                 |
      V                 V                 V
[ samtools_idx ]  [ samtools_stats ]  [ samtools_flagstat ]

\end{verbatim}
\end{document}